\documentclass[]{scrartcl}
\usepackage{url}
\usepackage{caption}
\usepackage{graphicx}
\graphicspath{  }
\usepackage{geometry}
\geometry{a4paper, portrait, margin=1in}
\usepackage{mathtools}



%opening
\title{Assignment 2: Part B}
\subtitle{16.888, Spring 2016}
\author{Sam Wald}
\date{}
\begin{document}

\maketitle

\paragraph{(b1) Simulation Implementation} \hfill \break
Unfortunately there are not many options for validating human space exploration mission elements. However, NASA has published a series of design reference architectures, or DRAs. NASA's DRA 5.0 presents four possible mission architectures using different propulsion methods. While our model is used to support a surface population as well as ferry crew and cargo between Mars and Earth, it can be used to model the DRA 5 architectures - a single round trip mission for crew of six - by fixing the transit population and surface populations at this minimum number so that no additional logistics are necessary beyond those crew. Other driving elements of the architecture are also defined within the model framework: 
\begin{center}
	\begin{tabular}{ lc lc}
Propulsion && NTR \\
TransitFuel.EARTH && LH$_2$\\
TransitFuel.EARTH && O$_2$\\
ReturnFuel.EARTH && LH$_2$\\
ReturnFuel.EARTH && O$_2$\\
Location && LEO\\
HabitatShielding && DEDICATED\\
ArrivalEntry && AEROCAPTURE\\
ArrivalCargoEntry && AEROCAPTURE\\
ArrivalDescent && AEROENTRY\\
Crew && 6\\
PowerSource && NUCLEAR\\
SurfaceCrew && 6\\
SurfaceShielding && DEDICATED\\
Site && GALE\\
FoodSource && EARTH ONLY\\
\end{tabular}
\end{center}
Running the model allows us to compare the results of each module output with the published values of element mass from DRA 5. A comparison of our model and DRA 5 are given below:
\begin{center}
	\begin{tabular}{ lc rc rc}
		\textbf{Element} && \textbf{Mass (kg)} && \textbf{\% off DRA5} \\\hline
		Transit Habitat && 44792 && 8\% \\
		Surface Habitat && 42258 && 79\% \\
		ECLSS && 19921 && 21\% \\
		Crew Transit Stage && 235269 && -29\% \\
		Cargo Transit Stage && 232034 && -6\% \\
		Ascent Vehicle && 44550 && 4\% \\
		\textbf{Total IMLEO} && \textbf{699339} && \textbf{-18\%} \\
	\end{tabular}
\end{center}
The model matches the mass of DRA to a high degree. However, there are some large errors, particularly in the surface habitat and cargo transit stage. These differences are due to the values of many parameters which go into to the model (listed in A1). Our models are based on published relations such as NASA's BVAD, but without additional details about subsystems in DRA5 or parameters used in its analysis, it is difficult to determine the sources of those differences. \hfill \break
Even with these differences, we feel that the model is useful for providing information on the relative impacts of changes to architectural variables. However, we are refining the modules to make sure that we understand the differences with DRA 5, especially in the areas of surface habitats, transit habitats, and surface power systems. This is important if we are to compare the utility of technology investments in subsystems which have coupled effects. 


\end{document}

